\apendice{Construção de Observáveis no \texttt{Qiskit}}
\label{ap:apendiceB}

Em mecânica quântica, um \textit{observável} é representado por um operador Hermitiano \(\hat{O}\), cujo valor esperado fornece a média dos resultados possíveis de uma medição sobre um estado quântico.

No contexto de computadores quânticos discretos (circuitos e \emph{qubits}), é mais conveniente trabalhar na base computacional formada pelos vetores \(\{\ket{x}\}\), com \(x\in\{0,\dots,2^n-1\}\). Nessa base, observáveis são frequentemente construídos a partir de combinações lineares de produtos de operadores de Pauli (\(I,X,Y,Z\)), formando a chamada base de Pauli. Assim, qualquer operador Hermitiano pode ser expresso como

\[
\hat{O} = \sum_i c_i P_i, \qquad P_i \in \{I,X,Y,Z\}^{\otimes n}, \quad c_i \in \mathbb{R}.
\]

Para medir a probabilidade de o circuito colapsar em um estado computacional específico \(\ket{x}\), utiliza-se o operador projetor correspondente:

\[
\hat{P}_x = \ket{x}\bra{x}.
\]

Em sua forma matricial, o projetor \(\hat{P}_x\) é uma matriz \(2^n\times2^n\) com zeros em todas as entradas exceto um elemento igual a 1 na posição diagonal correspondente ao índice binário de \(x\). Por exemplo, para quatro \emph{qubits} o estado \(\ket{1111}\) tem índice decimal 15, e \(\hat{P}_{1111}\) tem valor unitário apenas nessa entrada diagonal.

\begin{equation*}
P_{1111} = \begin{pmatrix}
    0 & 0 & \cdots & 0 \\
    0 & 0 & \cdots & 0 \\
    \vdots & \vdots & \ddots & \vdots \\
    0 & 0 & \cdots & 1
    \end{pmatrix}_{16\times16}
\end{equation*}

Essa representação matricial torna direta a implementação numérica em \textit{Qiskit} (por exemplo com \texttt{SparsePauliOp} ou \texttt{Operator}) e também facilita a leitura da relação entre valor esperado e probabilidade de medição quando se trabalha na base computacional.

Para um estado puro \(\ket{\psi}\) o valor esperado de um observável \(\hat{O}\) é dado por

\[
\braket{\hat{O}} = \braket{\psi | \hat{O} | \psi}.
\]

No caso mais geral de um estado misto descrito pela matriz densidade \(\rho\), usa-se a forma

\[
\langle \hat{O} \rangle = \mathrm{Tr}( \rho \hat{O} ).
\]

Em particular, para o projetor \(P_x = \ket{x}\bra{x}\) obtemos a probabilidade do resultado \(x\) na medição:

\[
\langle P_x \rangle = \mathrm{Tr}( \rho P_x ) = \bra{x} \rho \ket{x} = \mathrm{Prob}(x),
\]

que é a expressão discreta (base computacional) da regra de Born. Esta forma é a que se aplica diretamente ao tratamento de circuitos quânticos em computadores discretos e à interpretação dos valores retornados pela classe \texttt{Estimate} empregada neste trabalho.

No \textit{Qiskit}, observáveis construídos como combinações de Pauli podem ser representados e estimados via \texttt{SparsePauliOp} e \texttt{Operator}; para projetores individuais também é possível construir a matriz explícita (sparse) e avaliá-la sobre o estado final obtido pelo circuito. Ao implementar essas construções, atenção deve ser dada à ordenação de \emph{qubits} (endianness) e à correspondência lógica → física usada na transpila\c{c}\~ao para \textit{hardware} real, pois um mapeamento diferente muda a posição do bit alvo na representação matricial e, consequentemente, o índice do projetor.