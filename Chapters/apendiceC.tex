\apendice{Demonstração da Porta \textit{MCZ} a partir de \textit{MCX} e \textit{Hadamard}}
\label{ap:apendiceC}

O objetivo deste apêndice é demonstrar que a porta Z-multi-controlada ($MCZ$) pode ser construída a partir de uma porta X-multi-controlada ($MCX$) e duas portas \textit{Hadamard} (\(H\)). A sequência é dada por:

\begin{equation}
    MCZ = H~\text{MCX}~H
    \label{eq:MCZ}
\end{equation}

onde \(H\) é a porta Hadamard aplicada no qubit alvo da $MCX$.

\subsection*{Demonstração}

A porta $MCZ$ que intenta-se criar possui alvo no último \textit{qubit} apenas, sendo assim, a aplicação das $H$'s deve acontecer somente no último canal. Em termos de portas lógicas e matrizes, isso implica que a Equação~\ref{eq:MCZ}, na verdade, é

\begin{equation}
    MCZ_{16x16} = (I \otimes I \otimes I \otimes H)~MCX_{16x16}~(I \otimes I \otimes I \otimes H)
    \label{eq:MCZ 2}
\end{equation}

A matriz resultante da operação $(I \otimes I \otimes I \otimes H)$ é

\begin{equation}
    \bigl(~I^{\otimes 3}~ \otimes ~H~\bigr)_{16x16} = \frac{1}{\sqrt{2}}
    \begin{bmatrix}
        1 & 1 & 0 & 0 & \cdots & 0 & 0 \\
        1 & -1 & 0 & 0 & \cdots & 0 & 0 \\
        0 & 0 & 1 & 1 & \cdots & 0 & 0 \\
        0 & 0 & 1 & -1 & \cdots & 0 & 0 \\
        \vdots & \vdots & \vdots & \vdots & \ddots & 0 & 0 \\
        0 & 0 & 0 & 0 & 0 & 1 & 1 \\
        0 & 0 & 0 & 0 & 0 & 1 & -1 \\
    \end{bmatrix}
    _{16x16}
\end{equation}

Já a matriz de $MCX$ é

\begin{equation}
    MCX_{16x16} =
    \begin{bmatrix}
        1 & 0 & \cdots & 0 & 0 \\
        0 & 1 & \cdots & 0 & 0 \\
        \vdots & \vdots & \ddots & 0 & 0 \\
        0 & 0 & 0 & 0 & 1 \\
        0 & 0 & 0 & 1 & 0 
    \end{bmatrix}
    _{16x16}
\end{equation}

Voltando à Equação~\ref{eq:MCZ 2}, precisamos fazer o produto
As matrizes correspondentes a $X$, $Z$ e $H$ são, respectivamente

Para um sistema com múltiplos qubits, considere uma MCX com \(n-1\) qubits de controle e 1 qubit alvo:

\begin{equation}
    \text{MCX}\ket{c_1 c_2 \dots c_{n-1} t} = \ket{c_1 c_2 \dots c_{n-1}} \otimes X^{c_1 \cdots c_{n-1}}\ket{t}
\end{equation}

Aplicando Hadamard antes e depois da MCX no qubit alvo:

\begin{align*}
    H \cdot MCX \cdot H \ket{c_1 c_2 \dots c_{n-1} t} 
    &= H \cdot MCX \bigl( \ket{c_1 c_2 \dots c_{n-1}} \otimes H\ket{t} \bigr) \\
    &= H \bigl( \ket{c_1 c_2 \dots c_{n-1}} \otimes X^{c_1 \cdots c_{n-1}} H\ket{t} \bigr) \\
    &= \ket{c_1 c_2 \dots c_{n-1}} \otimes \bigl( H X^{c_1 \cdots c_{n-1}} H \bigr) \ket{t} \\
    &= \ket{c_1 c_2 \dots c_{n-1}} \otimes Z^{c_1 \cdots c_{n-1}} \ket{t} \\
    &= MCZ \ket{c_1 c_2 \dots c_{n-1} t}
\end{align*}

Portanto, a sequência \(H \cdot MCX \cdot H\) no qubit alvo é equivalente à porta MCZ.

\subsection*{Representação em Qiskit}

No Qiskit, podemos implementar a $MCZ$ usando a $MCX$ e portas Hadamard como segue:
\begin{figure}
    \centering
    \caption{Implementação de porta $MCZ$ a partir de $MCX$ e $H$'s}
    \label{cod:apendiceC}
    \begin{lstlisting}{python}
        from qiskit import QuantumCircuit
        
        qc = QuantumCircuit(4)  # 3 controles + 1 alvo
        qc.h(3)                 # Hadamard no qubit alvo
        qc.mcx([0,1,2], 3)      # MCX com controles 0, 1 e 2 e alvo 3
        qc.h(3)                 # Hadamard no qubit alvo
        qc.draw('mpl')
    \end{lstlisting}
\end{figure}
