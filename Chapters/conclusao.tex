\chapter{CONCLUSÃO}
\label{sec:conclusao}

O presente trabalho de conclusão de curso investigou, de forma empírica e rigorosa, a aplicação e a eficácia de métodos de mitigação e supressão de ruídos quânticos em processadores de \emph{hardware} real (QPUs) da arquitetura IBM Quantum, tendo o Algoritmo de Grover como principal objeto de teste em regime NISQ (\emph{Noisy Intermediate-Scale Quantum}). A pesquisa evidenciou conquistas significativas ao demonstrar que a mitigação de ruído não é apenas um complemento, mas uma necessidade crítica para extrair valor de \emph{backends} atuais. 

A principal conquista reside na validação de que a combinação seletiva e complementar de técnicas supera o desempenho das técnicas isoladas, identificando a sinergia entre o \emph{Zero-Noise Extrapolation} (ZNE) e o \emph{Probabilistic Error Amplification} (PEA) como a estratégia mais robusta para o circuito de Grover, elevando a fidelidade do estado-alvo para um patamar superior ao obtido em execuções sem mitigação. Em essência, o estudo comprova que, ao invés da acumulação indiscriminada de métodos, a escolha criteriosa de técnicas que se compensam mutuamente é o caminho mais eficaz para o aumento da fidelidade, alinhando os resultados práticos com as considerações teóricas da otimização de circuitos.

Contudo, o estudo também revelou limitações importantes, fornecendo reconsiderações fundamentais. O desempenho abaixo do esperado da combinação máxima de técnicas (incluindo \emph{Dynamical Decoupling} -- DD) demonstrou que a sobreposição de efeitos de mitigação e o aumento excessivo da profundidade do circuito podem, na verdade, degradar a fidelidade geral. Este fato se relaciona diretamente à teoria que suporta o DD, cuja ineficácia foi verificada em circuitos ativos e densos como o de Grover, onde a falta de ociosidade de \emph{qubits} faz com que a sobrecarga dos pulsos de DD gere mais ruído do que suprime, validando o princípio físico de sua aplicação limitada.

Além disso, a comparação entre diferentes \emph{backends} (\texttt{ibm\_torino} e \texttt{ibm\_brisbane}) reforçou a crucialidade da dependência do \emph{hardware}, confirmando que a variabilidade do ruído intrínseco de cada QPU é um fator não negligenciável na obtenção de resultados confiáveis. 

Como sugestão para trabalhos futuros, é imperativo investigar a otimização de métodos de extrapolação mais avançados dentro da combinação ZNE - PEA, bem como expandir a análise para QPUs baseadas em tecnologias alternativas (por exemplo, íons presos), verificando a universalidade das conclusões obtidas. Por fim, aprofundar a pesquisa em soluções \emph{hardware-aware}, como sequências de DD adaptadas ao ruído e à topologia do circuito de Grover, pode pavimentar o caminho para a mitigação de ruído na fronteira entre a mitigação e a correção de erros, solidificando o presente trabalho como uma contribuição valiosa para o desenvolvimento de aplicações quânticas na atual era tecnológica.