\chapter{INTRODUÇ\~{A}O}
\label{chap: introducao}

No início do desenvolvimento dos primeiros computadores, com a \emph{M\'{a}quina de Turing} em 1936 e, posteriormente, a criaç\~{a}o dos transistores nos Laboratórios da \emph{Bell Telephone}, em 1947 \cite{transistor_hist}, e sua evoluç\~{a}o at\'{e} passar a ser utilizado na construç\~{a}o de processadores, muito j\'{a} se pensava a respeito dos incríveis feitos que essa emergente tecnologia poderia proporcionar, e realmente podemos ver tal expectativa sendo concretizada nos dias atuais, com os computadores sendo acessíveis at\'{e} mesmo da palma da m\~{a}o, com os \textit{smartphones}, algo inimagin\'{a}vel àquela \'{e}poca.

Contudo, embora ainda seja – e continuar\~{a}o sendo – demasiadamente útil, existem novos problemas que exigem uma forma diferente de processamento, como Richard Feynman afirmou em 1981 na primeira Confer\^{e}ncia da Física da Computaç\~{a}o, na qual sugeriu que computadores qu\^{a}nticos poderiam realizar simulações que s\~{a}o impossíveis de serem realizadas pelos computadores cl\'{a}ssicos, mesmo aqueles mais robustos. Isso se deve à natureza intrínseca aos próprios problemas, como, por exemplo, a simulaç\~{a}o de mol\'{e}culas grandes ou materiais complexos, que exigem um poder de processamento que cresce exponencialmente com o aumento do número de partículas. Em adiç\~{a}o, Feynman tamb\'{e}m descreveu o mundo físico como “qu\^{a}ntico”, e como tal, sistemas físicos (qu\^{a}nticos) apenas seriam adequadamente simulados por meio de computadores que utilizassem princípios qu\^{a}nticos de processamento \cite{feynman1982}.

A partir desse ponto, muitos adeptos e fomentadores da ideia contribuíram com conhecimento, como David Deutsch, que propôs o primeiro algoritmo qu\^{a}ntico, mesmo que com aplicações limitadas, foi demonstrada uma efici\^{e}ncia muito superior aos algoritmos cl\'{a}ssicos \cite{deutsch1985}, al\'{e}m de servir de suporte para o desenvolvimento de outros algoritmos importantes, como os Algoritmos de Simon \cite{simon1994} e de Shor . Tamb\'{e}m Peter Shor mostrou com seu algoritmo qu\^{a}ntico de fatoraç\~{a}o que um algoritmo qu\^{a}ntico poderia fatorar números inteiros exponencialmente mais r\'{a}pido que algoritmos cl\'{a}ssicos \cite{shor1994}. E, ainda, Lov Grover, que apresentou um algoritmo de busca em listas desordenadas, que opera por meio de uma t\'{e}cnica de amplificaç\~{a}o tanto de amplitudes quanto das probabilidades e que possui melhora quadr\'{a}tica na efici\^{e}ncia em relaç\~{a}o a m\'{e}todos cl\'{a}ssicos conhecidos \cite{grover1996}. 

Seguindo o vi\'{e}s j\'{a} apresentado, este trabalho tem por objetivo se envolver no estudo, construç\~{a}o, melhoria de resultados e an\'{a}lise do Algoritmo de Grover aplicado a um sistema de quatro qubits, a fim de melhor compreend\^{e}-lo, e ainda conseguir gerar conteúdo que poder\'{a} ser base para novos trabalhos. As melhorias propostas neste trabalho est\~{a}o relacionadas aos m\'{e}todos de supress\~{a}o – que \'{e} uma estrat\'{e}gia preventiva, aplicada durante a execuç\~{a}o do circuito qu\^{a}ntico para minimizar a geraç\~{a}o de erros – e de mitigaç\~{a}o – ocorre após a execuç\~{a}o do circuito qu\^{a}ntico, com t\'{e}cnicas que tentam corrigir ou compensar os erros nos resultados obtidos – de ruídos qu\^{a}nticos gerados pelas \textit{Quantum Processing Unit} (QPUs) nos Computadores Qu\^{a}nticos (CQs). Ambas estrat\'{e}gias possuem diversos m\'{e}todos incluídos no \emph{Qiskit}, alguns deles ser\~{a}o empregados, suas influ\^{e}ncias na qualidade final dos resultados ser\~{a}o analisadas e discutidas no discorrer deste texto.

% O objetivo deste documento \'{e} apresentar o uso b\'{a}sico da classe {\tt uflamon} para a elaboraç\~{a}o de monografias da UFLA utilizando a linguagem de marcaç\~{a}o \LaTeX\ \cite{Lamport1994}.  A maioria dos comandos (macros) e ambientes das classes b\'{a}sicas da linguagem \'{e} v\'{a}lida tamb\'{e}m nessa classe, que \'{e} estendida com comandos para confecç\~{a}o da capa, p\'{a}ginas de rosto, dedicatórias, etc.

% A classe foi baseada inicialmente nas normas da PRPG/UFLA para produç\~{a}o de TCC \cite{PRPG2006}. Essas normas foram posteriormente atualizadas, de maneira geral pela UFLA, para a produç\~{a}o de monografias, dissertações e teses \cite{BIB2010}.  A vers\~{a}o atual da {\tt uflamon} reflete a última vers\~{a}o da norma \cite{UFLA:2015}.

% Este texto, que objetiva apresentar um exemplo de uso da classe  {\tt uflamon}, encontra-se organizado como se segue. O Capítulo~\ref{cap: elementos} apresenta exemplos de inserç\~{a}o de figuras, tabelas, equações e demais elementos explicativos. O Capítulo~\ref{cap: conclusao} apresenta coment\'{a}rios e observações finais. Por fim, o Ap\^{e}ndice~\ref{cap: apendice} mostra como elaborar um ap\^{e}ndice simples.