\section{Visão Geral do Algoritmo de Grover}
\label{sec:algGroverGeral}

Como j\'{a} mencionado, o Algoritmo de Grover apresenta um m\'{e}todo de busca em listas que n\~{a}o possuem qualquer tipo de ordenaç\~{a}o com uma efici\^{e}ncia de ordem quadr\'{a}tica, por meio de uma t\'{e}cnica de amplificaç\~{a}o tanto de amplitudes quanto das probabilidades.
Seja a ilustraç\~{a}o apresentada na Figura~\ref{fig:itemMarcado},

\begin{figure}[ht!]
    \centering
    \caption{Lista com $2^{n}$ itens e um elemento $\omega$ marcado.}
    \includegraphics[scale=.6]{Imagens/markedItem.png}\\
    {\small Fonte: Github/\emph{Qiskit}}
    \label{fig:itemMarcado}
\end{figure}

Aqui tem-se ilustrado um conjunto contendo $N = 2^n$ elementos organizados aleatoriamente. Nele, h\'{a} um termo $\omega$ destacado, nomeado "\textit{winner}", que \'{e} justamente o elemento procurado. Se quisesse-se buscar o item marcado por meio de computaç\~{a}o cl\'{a}ssica, seriam necess\'{a}rias, em m\'{e}dia, $\frac{N}{2}$ iterações, contudo, ao implementarmos o Algoritmo de Grover, esse valor cai para $\sqrt{N}$, que \'{e} um avanço significativo, principalmente quando considera-se manipular conjuntos com grandes quantidades de itens \cite{qiskit_GroverNotebook}.

\subsection{Estrutura}
\label{subsec:estruturaAlg}

Para que seja poss\'{i}vel compreender a atuaç\~{a}o do algoritmo, \'{e} preciso entender sua estrutura, que conta com tr\^{e}s partes necess\'{a}rias para que possa ser implementado, sendo elas: preparaç\~{a}o inicial do estado, determinaç\~{a}o do Or\'{a}culo (marcaç\~{a}o do estado buscado) e amplificaç\~{a}o da amplitude e probabilidade (aplicaç\~{a}o do Operador de Grover ou Operador de Difus\~{a}o). Isso pode ser exemplificado pelo esquema da Figura~\ref{fig:groverEsquema}.

\begin{figure}[ht!]
    \centering
    % \captionsetup{justification=centering}
    \caption{Esquematizaç\~{a}o da estrutura do Algoritmo de Grover.}
    \includegraphics[width=.7\textwidth]{Imagens/esquemaG.png}\\
    {\small Fonte: Github/\emph{Qiskit}.}
    \label{fig:groverEsquema}
\end{figure}

Na preparaç\~{a}o do estado h\'{a} a criaç\~{a}o do que chamaremos de espaço de pesquisa, \'{e} o conjunto que cont\'{e}m o elemento marcado e onde ele ser\'{a} procurado. 

Feito isso, seguiremos para a determinaç\~{a}o do item que queremos encontrar, que \'{e} feita pelo Or\'{a}culo. A atuaç\~{a}o do Or\'{a}culo consiste em inverter a fase do elemento de interesse, ou seja, ele marca o item buscado. Caso haja mais de um item de interesse, o Or\'{a}culo deve agir em todos eles, marcando-os, de modo que sejam destacados pelo difusor.

Por fim, utiliza-se o Operador de Difus\~{a}o para aumentar a amplitude do estado marcado, enquanto decresce as demais, sendo assim uma garantia de que na medida final, o resultado obtido seja aquele procurado.

% \subsection{Funcionamento}
% \label{subsec:funcionamentoAlg}
Quando trabalha-se com o Algoritmo de Grover, pode-se reduzir o problema a um espaço de duas dimensões, sendo necess\'{a}rio considerar apenas o estado buscado -- \textit{winner}, $\ket{\omega}$ --, e a superposiç\~{a}o uniforme, $\ket{s}$. Contudo, esses n\~{a}o s\~{a}o vetores perpendiculares entre si, uma vez que $\ket{\omega}$ ocorre em superposiç\~{a}o com amplitude $\frac{1}{\sqrt{N}}$. Para contornar isso, cria-se um vetor $\ket{s'}$ que \'{e} perpendicular a $\ket{\omega}$, formando assim o plano bidimensional apresentado na Figura~\ref{subFig:Plano}. Na Figura~\ref{subFig:Amplitude}, est\~{a}o ilustradas as amplitudes dos itens contidos no conjunto, evidenciando o elemento \textit{winner}.

\begin{figure}[ht!]
    \centering
    \caption{Plano Bidimensional e Amplitude dos estados da base.}
    \label{fig:espacoAmplitude}
    \begin{subfigure}[b]{0.4\textwidth}
        \centering
        \includegraphics[width=\textwidth]{Imagens/BidimensionalEspaco.png}
        \caption{Plano formado pelos vetores $\ket{\omega}$ e $\ket{s'}$.}
        \label{subFig:Plano}
    \end{subfigure}
    \hfill
    \begin{subfigure}[b]{0.45\textwidth}
        \centering
        \includegraphics[width=\textwidth]{Imagens/AmplitudesW.png}
        \caption{Amplitude dos itens do conjunto, com $\ket{\omega}$ em destaque.}
        \label{subFig:Amplitude}
    \end{subfigure}

    \vspace{0.3em}
    {\small Fonte: Github/\emph{Qiskit}.}
\end{figure}


% Uma vez introduzida a vis\~{a}o geral sobre como o algoritmo atua, pode-se passar para o estudo de cada uma de suas tr\^{e}s etapas.

%%%%%%%%%%%%%%%%%%%%%%%%%%%%%%%%%%%%%%%%%%%%%%%%%%%%%%%%%%%%%%%%%%%%%%%%%%%%%%%%%%

\subsection{Preparaç\~{a}o Inicial}
\label{subsec:prepInicialAlg}

A Preparaç\~{a}o Inicial \'{e} a etapa inicial do algoritmo, na qual o Espaço de Pesquisa \'{e} criado, matematicamente descrito como uma superposiç\~{a}o uniforme e pela Equaç\~{a}o~\ref{eq:superposicao}, e possui tamanho dado por $N = 2^n$, com $n$ sendo o n\'{u}mero de qubits.

\begin{equation}
    \ket{s} = \frac{1}{\sqrt{N}}\sum_{x = 0}^{N - 1} \ket{bin(x)},
    \label{eq:superposicao}
\end{equation}
%
com $x \in \left\{ 0, 1, 2, \ldots, N-1 \right\}$.

Se uma mediç\~{a}o for realizada na base $\ket{x}$, de acordo com o Quinto Postulado da Mec\^{a}nica Qu\^{a}ntica \cite{vonNuemann1955_QuMec}, a superposiç\~{a}o colapsa, podendo resultar em qualquer um dos estados de mesma probabilidade $\frac{1}{N} = \frac{1}{2^n}$. 

Em termos de circuito qu\^{a}ntico, a superposiç\~{a}o explicitada pela Equaç\~{a}o~\ref{eq:superposicao} pode ser constru\'{i}da facilmente aplicando a porta Hadamard em cada um dos qubits, que se iniciam no estado fundamental $\ket{0}$, como representado pela Equaç\~{a}o~\ref{eq:preparacaoInicial}.
%
\begin{equation}
    \ket{s} = H^{\otimes n}~\ket{0}^n
    \label{eq:preparacaoInicial}
\end{equation}

O vetor $\ket{s}$ que aparece na Figura~\ref{subFig:Plano} pode ser escrito em coordenadas polares, como
%
\begin{equation}
    \ket{s} = \sin{\theta}\ket{\omega} + \cos{\theta}\ket{s'},
    \label{eq:estadoCoordPolar}
\end{equation}

em que 
%
\begin{equation}
    \theta = \arcsin{\braket{s|\omega}} = \arcsin{\frac{1}{\sqrt{N}}}
    \label{eq:thetaValue}
\end{equation}

\subsection{Or\'{a}culo ($U_f$)}
\label{subSec:oraculoAlg}

Tomando a premissa de que o elemento $\omega$ procurado está em um determinado conjunto contendo $N = 2^n$ itens, ${0, 1, 2, ..., N-1}$, com $n \in \mathbb{N}$, podemos recorrer a uma função $f : {0, 1, 2, ..., N-1},\\ \to {1,-1}$ que atua na amplitude dos elementos para marcar o buscado, sendo $f$ tal que
%
\begin{equation}
    f(x) = 
    \begin{cases}
        -1, & \text{se } x = \omega \\
        1, & \text{se } x \neq \omega
    \end{cases}
    \label{eq:fx Oraculo}
\end{equation}

Em outras palavras, o Oráculo, denotado por \simboloinline{$U_f$}{Operador Oráculo}, nada mais é que uma função que gera a reflexão em um \^{a}ngulo $\theta$ de $\ket{s}$ sobre $\ket{s'}$, bem como a reflexão da amplitude do elemento de interesse $\ket{\omega}$, que pode ser visualizado com a ajuda da Figura~\ref{fig:rotacaoReflexao}.

\begin{figure}[ht!]
    \centering
    \captionsetup{justification=centering}
    \caption{Reflexão dos estados $\ket{s}$ e $\ket{\omega}$, respectivamente.}

    \begin{subfigure}[b]{0.4\textwidth}
        \centering
        \includegraphics[width=\textwidth]{Imagens/rotacao.png}
        \caption{Reflexão de $\ket{s}$ sobre $\ket{s'}$ em um \^{a}ngulo $\theta$.}
        \label{subFig:rotacao}
    \end{subfigure}
    \hfill
    \begin{subfigure}[b]{0.45\textwidth}
        \centering
        \includegraphics[width=\textwidth]{Imagens/reflexao.png}
        \caption{Reflexão de $\pi$ rads na amplitude de $\ket{\omega}$.}
        \label{subFig:reflexao}
    \end{subfigure}

    \vspace{0.5em}
    {\small Fonte: Github/\emph{Qiskit}.}
    \label{fig:rotacaoReflexao}
\end{figure}


De modo geral, a construção dos Oráculos no Algoritmo de Grover pode ser realizada utilizando portas lógicas qu\^{a}nticas do tipo $X$ e \simboloinline{$MCZ$}{\textit{Porta Lógica Quântica Z-multi-controlada}}, seguindo a seguinte regra de formação:

\begin{enumerate}
\label{enum: regraFormacaoOraculo}
    \item Para cada qubit cujo valor no estado buscado seja $\ket{0}$, aplica-se uma porta $X$ nesse qubit, realizando assim um \textit{bit-flip};
    
    \item Em seguida, aplica-se uma porta ($MCZ$) sobre todos os qubits, marcando o estado desejado com um fator de fase;
    
    \item Por fim, aplica-se novamente a porta $X$ nos mesmos qubits que sofreram o \textit{bit-flip} na primeira etapa, revertendo as alterações realizadas inicialmente.
\end{enumerate}

\subsection{Operador de Difus\~{a}o (\(U_s\))}
\label{subSec:difusaoAlg}

A  aplicação do Operador de Difusão, denotado por \simboloinline{$U_s$}{\textit{Operador de Difus\~{a}o}}, consiste em operações de amplificação sobre o estado $\ket{s}$, gerando uma reflexão adicional do mesmo.
%
\begin{equation}
    U_s = 2~\ket{s}\bra{s} - I
    \label{eq:Us}
\end{equation}

Essa transformação mapeia o estado de $\ket{s}$ para $(U_sU_f)\ket{s}$ e completa a transformação, tal como está apresentado na Figura~\ref{fig:transformacaoCompleta}.

\begin{figure}[ht!]
    \centering
    \captionsetup{justification=centering}
    \caption{Segunda reflexão dos estados $\ket{s}$ e $\ket{\omega}$, respectivamente.}
    \label{fig:transformacaoCompleta}  % Label principal após caption

    \begin{subfigure}[b]{0.4\textwidth}
        \centering
        \includegraphics[width=\textwidth]{Imagens/reflexaoEstado.png}
        \caption{Segunda reflexão de $\ket{s}$ sobre $\ket{s'}$.}
        \label{subFig:reflexaoEstado}
    \end{subfigure}
    \hfill
    \begin{subfigure}[b]{0.48\textwidth}
        \centering
        \includegraphics[width=\textwidth]{Imagens/reflexaoAmplitude.png}
        \caption{Segunda reflexão na amplitude de $\ket{\omega}$.}
        \label{subFig:reflexaoAmplitude}
    \end{subfigure}

    \vspace{0.5em}
    {\small Fonte: Github/\emph{Qiskit}.}
\end{figure}

Duas reflexões sempre resultam em uma rotação, que leva o estado inicial $\ket{s}$ para mais perto do elemento \textit{winner}, $\ket{\omega}$. 

Como esta é uma reflexão sobre $\ket{s}$, busca-se adicionar uma fase negativa a cada estado ortogonal a $\ket{s}$. A maneira como pode-se fazer isso é aplicar uma transforma\~{a}o que leva o estado $\ket{s} \to \ket{0}$, com isso, em vez de $U_s$, chamar-se-á $U_0$. Dessa forma, a Equaç\~{a}o~\ref{eq:Us} se torna: 
%
\begin{equation}
    U_0 = 2~\ket{0}\bra{0} - I
    \label{eq:Uo}
\end{equation}

Essa operação aplica uma fase negativa a todos os estados ortonormais a $\ket{0}$, como será demonstrado.

% [TALVEZ COLOCAR NUM APÊNDICE***]

Aplicação do Operador $U_0$ em $\ket{s}$, com $\ket{s}$ podendo ser um conjunto arbitr\'{a}rio com $N-1$ itens:

\begin{equation*}
U_0~\ket{s} = \left(2\ket{0}\bra{0} - I\right) \frac{1}{\sqrt{N}}\sum_{x=0}^{N-1} \ket{bin(x)},
\end{equation*}

Expandindo, tem-se:

\begin{equation*}
2~\ket{0}\bra{0} \left( \frac{1}{\sqrt{N}}\sum_{x=0}^{N-1} \ket{bin(x)}\right) = \frac{1}{\sqrt{N}}\sum_{x=0}^{N-1} \ket{bin(x)}
\end{equation*}

\begin{equation}
2~\ket{0} \left( \sum_{x=0}^{N-1} \braket{0|bin(x)}\right) = \sum_{x=0}^{N-1} \ket{bin(x)}
\label{eq:aplicacaoDifusor 2}
\end{equation}
%
Como $\braket{0|bin(x)} = 1$ apenas se $x = 0$ (enquanto todos os outros termos se tornam $0$),  o único termo que sobrevive é:
%
\begin{equation*}
    2\ket{0}{\braket{0|bin(0)}} = 2\ket{0}\braket{0|0} = 2\ket{0} = 2\ket{bin(0)}
\end{equation*}

Então, em síntese, tem-se:
%
\begin{align}
\notag
U_0~\ket{s} = &2~\ket{bin(0)} - \sum_{x=0}^{N-1} \ket{bin(x)} \\ 
\notag
= &2~\ket{bin(0)} - \ket{bin(0)} - \ket{bin(1)} - \ket{bin(2)} - \dots - \ket{bin(N-1)} \\ 
= &\ket{bin(0)} - \sum_{x=1}^{N-1} \ket{bin(x)}
\label{eq:aplicacaoDifusao}
\end{align}
%
Ou seja, a amplitude $\ket{bin(0)}$ aumenta em relação aos demais que invertem.

Tendo sido demonstrada a forma matemática, passa-se agora para a apresentação da obtenção do circuito qu\^{a}ntico que realiza essa operação. 

O Operador de Difusão em termos de portas lógicas é dado por:
%
\begin{equation}
    U_s = H^{\otimes n}~U_0~H^{\otimes n}
    \label{gate: operadorDifusao}
\end{equation}

Como já visto, a atuação de $U_0$ é aplicar uma fase negativa a todos os estados ortogonais a $\ket{s}$, e, para isso, deve ser realizada a seguinte sequência:

\begin{enumerate}
\label{enum: difusor}
    \item Aplicação de portas $X$ em todos canais para \textit{bit-flip} dos estados;
    \[\ket{00\cdots0} \to \ket{11\cdots1} \]
    \item Aplicação de porta $MCZ$, com alvo no último canal, pois dessa forma o último estado recebe uma inversão de fase;
    \begin{equation}
    MCZ_{[N \times N]}  =
    \begin{bmatrix}
        1 & 0 &  \cdots  & 0\\
        0 & 1 &  \cdots  & 0\\
        \vdots & \vdots & \ddots & \vdots \\
        0 & 0 &  \cdots & -1\\
    \end{bmatrix}_{[N \times N]}
    \label{mtx: gateMCZ}
\end{equation}
    \item Novamente aplica-se portas $X$ para desfazer os \textit{bit-flips}.
    \[\ket{11\cdots1} \to \ket{00\cdots0} \]
\end{enumerate}

Após a última etapa, o resultado obtido é:

\begin{equation*}
    \begin{bmatrix}
        -1 & 0 &  \cdots  & 0\\
        0 & 1 &  \cdots  & 0\\
        \vdots & \vdots & \ddots & \vdots \\
        0 & 0 &  \cdots & 1\\
    \end{bmatrix}_{[N \times N]}
\end{equation*}

Ou seja, existe ainda uma fase global em $U_0$, de modo que o Operador de Difusão, em termos de portas lógicas, é:
\begin{equation}
    U_0 = -X^{\otimes n}~MCZ~X^{\otimes n}
    \label{gate: difusor}
\end{equation}

O trecho de circuito descrito em Equação~\ref{gate: difusor} está mostrado na Figura~\ref{fig:difusor}

\begin{figure}[!htb]
\centering
\caption{Operador de Difus\~{a}o parcial}
\label{fig:difusor} 
\begin{quantikz}
\lstick{$q_0$}       & \gate{X} & \qw      & \ctrl{1}   & \qw       & \gate{X}  & \qw \\
\lstick{$q_1$}       & \gate{X} & \qw      & \ctrl{2}   & \qw       & \gate{X}  & \qw \\
\lstick{$\vdots$}    &~\vdots~  &          &            & \qw       &~\vdots~   & \qw \\
\lstick{$q_{n-1}$}   & \gate{X} & \qw      & \gate{Z}   & \qw       & \gate{X}  & \qw \\
\lstick{$c: n$}      & \cw      & \cw      & \cw        & \cw       & \cw       & \cw
\end{quantikz}

\vspace{.3em}
{\small Fonte: do autor} 
\end{figure}

Considerando as Equações~\ref{gate: operadorDifusao} e~\ref{gate: difusor}. Tem-se que o Operador de Difusão completo em portas lógicas qu\^{a}nticas são dados pela Equação~\ref{gate: operadorDifusaoCompleto} e pela Figura~\ref{fig:operadorDifusaoCompleto}.

\begin{equation}
    U_s = H^{\otimes n}~X^{\otimes n}~MCZ~X^{\otimes n}~H^{\otimes n}
    \label{gate: operadorDifusaoCompleto}
\end{equation}

Note que a fase negativa não foi acrescentada, pois não precisa ser considerada no circuito qu\^{a}ntico. 

\begin{figure}[!htb]
\centering
\caption{Operador de Difus\~{a}o completo}
\label{fig:operadorDifusaoCompleto} %rotulo para refencia
\begin{quantikz}
\lstick{$q_0$}       & \gate{H} & \gate{X} &          & \ctrl{1} &          & \gate{X} & \gate{H} & \qw \\
\lstick{$q_1$}       & \gate{H} & \gate{X} &          & \ctrl{2} &          & \gate{X} & \gate{H} & \qw \\
\lstick{$\vdots$}    &~\vdots~  &~\vdots~  &          &          &          & ~\vdots~ & ~\vdots~ & \qw \\
\lstick{$q_{n-1}$}   & \gate{H} & \gate{X} & \qw      & \gate{Z} & \qw      & \gate{X} & \gate{H} & \qw \\
\lstick{$c: n$}      & \cw      & \cw      & \cw      & \cw      & \cw      & \cw      & \cw      & \cw
\end{quantikz}

\vspace{.3em}
{\small Fonte: do autor} 
\end{figure}

\subsection{Fator de Otimizac\~{a}o (k)}
\label{subsec:otimizacao k}

Afim de otimizar o resultado, a aplicação do \nameref{subSec:oraculoAlg} e do \nameref{subSec:difusaoAlg} deverão ser executados um número \simboloinline{$k$}{Fator de Otimização do Algoritmo de Grover} de vezes para que o resultado obtido seja o mais próximo possível (com maiores amplitudes e probabilidades) do estado esperado $\ket{\omega}$, e pode ser expresso por
%
\begin{equation}
    \label{eq:psi k}
    \ket{\psi_k} = (U_sU_f)^k\ket{s}
\end{equation}

O número ideal $k$ de iterações necessárias para que $\ket{\omega}$ seja obtido é dado por

\begin{equation}
    k = \frac{\pi}{4}\sqrt{\frac{N}{m}}
    \label{eq:k value}
\end{equation}
%
em que $N$ é o tamanho do espaço de pesquisa, e $m$ é a quantidade de itens procurados.

A Figura~\ref{fig:algoritmoCompleto} mostra um esquema do Algoritmo de Grover completo, evidenciando o Oráculo e o Operador de Difusão.

\begin{figure}[ht!]
    \centering
    \captionsetup{justification=centering}
    \caption{Esquematização do Algoritmo de Grover.}
    \label{fig:algoritmoCompleto}
    
    \includegraphics[scale=.5]{Imagens/esquemaAlgG.png}
    
    {\small Fonte: Github/\emph{Qiskit}.}
\end{figure}
